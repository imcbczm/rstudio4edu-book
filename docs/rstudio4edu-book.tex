\documentclass[]{book}
\usepackage{lmodern}
\usepackage{amssymb,amsmath}
\usepackage{ifxetex,ifluatex}
\usepackage{fixltx2e} % provides \textsubscript
\ifnum 0\ifxetex 1\fi\ifluatex 1\fi=0 % if pdftex
  \usepackage[T1]{fontenc}
  \usepackage[utf8]{inputenc}
\else % if luatex or xelatex
  \ifxetex
    \usepackage{mathspec}
  \else
    \usepackage{fontspec}
  \fi
  \defaultfontfeatures{Ligatures=TeX,Scale=MatchLowercase}
\fi
% use upquote if available, for straight quotes in verbatim environments
\IfFileExists{upquote.sty}{\usepackage{upquote}}{}
% use microtype if available
\IfFileExists{microtype.sty}{%
\usepackage{microtype}
\UseMicrotypeSet[protrusion]{basicmath} % disable protrusion for tt fonts
}{}
\usepackage{hyperref}
\hypersetup{unicode=true,
            pdftitle={rstudio4edu Book},
            pdfauthor={Alison Hill; Desirée De Leon},
            pdfborder={0 0 0},
            breaklinks=true}
\urlstyle{same}  % don't use monospace font for urls
\usepackage{natbib}
\bibliographystyle{apalike}
\usepackage{longtable,booktabs}
\usepackage{graphicx,grffile}
\makeatletter
\def\maxwidth{\ifdim\Gin@nat@width>\linewidth\linewidth\else\Gin@nat@width\fi}
\def\maxheight{\ifdim\Gin@nat@height>\textheight\textheight\else\Gin@nat@height\fi}
\makeatother
% Scale images if necessary, so that they will not overflow the page
% margins by default, and it is still possible to overwrite the defaults
% using explicit options in \includegraphics[width, height, ...]{}
\setkeys{Gin}{width=\maxwidth,height=\maxheight,keepaspectratio}
\IfFileExists{parskip.sty}{%
\usepackage{parskip}
}{% else
\setlength{\parindent}{0pt}
\setlength{\parskip}{6pt plus 2pt minus 1pt}
}
\setlength{\emergencystretch}{3em}  % prevent overfull lines
\providecommand{\tightlist}{%
  \setlength{\itemsep}{0pt}\setlength{\parskip}{0pt}}
\setcounter{secnumdepth}{5}
% Redefines (sub)paragraphs to behave more like sections
\ifx\paragraph\undefined\else
\let\oldparagraph\paragraph
\renewcommand{\paragraph}[1]{\oldparagraph{#1}\mbox{}}
\fi
\ifx\subparagraph\undefined\else
\let\oldsubparagraph\subparagraph
\renewcommand{\subparagraph}[1]{\oldsubparagraph{#1}\mbox{}}
\fi

%%% Use protect on footnotes to avoid problems with footnotes in titles
\let\rmarkdownfootnote\footnote%
\def\footnote{\protect\rmarkdownfootnote}

%%% Change title format to be more compact
\usepackage{titling}

% Create subtitle command for use in maketitle
\providecommand{\subtitle}[1]{
  \posttitle{
    \begin{center}\large#1\end{center}
    }
}

\setlength{\droptitle}{-2em}

  \title{rstudio4edu Book}
    \pretitle{\vspace{\droptitle}\centering\huge}
  \posttitle{\par}
    \author{Alison Hill \\ Desirée De Leon}
    \preauthor{\centering\large\emph}
  \postauthor{\par}
      \predate{\centering\large\emph}
  \postdate{\par}
    \date{2019-06-20}

\usepackage{booktabs}

\begin{document}
\maketitle

{
\setcounter{tocdepth}{1}
\tableofcontents
}
\hypertarget{prerequisites}{%
\chapter*{Prerequisites}\label{prerequisites}}
\addcontentsline{toc}{chapter}{Prerequisites}

To compile this example to PDF, you need XeLaTeX. You are recommended to install TinyTeX (which includes XeLaTeX): \url{https://yihui.name/tinytex/}.

\^{} Is this code chunk important?

\hypertarget{intro}{%
\chapter{Introduction}\label{intro}}

\hypertarget{learner-personas}{%
\chapter{Learner Personas}\label{learner-personas}}

Here are the learner personas we are keeping in mind.

\hypertarget{teaching-and-learning-with-jupyter-book-review}{%
\chapter{Teaching and Learning with Jupyter Book Review}\label{teaching-and-learning-with-jupyter-book-review}}

\textbf{Good stuff}

\begin{itemize}
\tightlist
\item
  Description of the use cases
\item
  Screenshots of examples of activities/questions in notebooks (e.g.~in Section \href{https://jupyter4edu.github.io/jupyter-edu-book/why-we-use-jupyter-notebooks.html\#why-do-we-use-jupyter}{2}.3.22.
\item
  Inclusion of `pro-tips content' throughout content
\item
  I do like the pedagogical patterns section, though I see how including something like this greatly increases the scope of our project. I think what I like most about it, is that they're concrete examples of creative use-cases.
\item
  The \href{https://jupyter4edu.github.io/jupyter-edu-book/jupyter.html\#tips-and-tricks}{tips and tricks section} I find to be useful, practical advice. In thinking of how this applies toour own project, I think a lot of this could be condensed into little ``tips'' boxes.
\item
  Different ways of running Jupyter w/ pros and cons. Useful for educators to know ahead of time
\end{itemize}

\textbf{Stuff I don't like as much}

\begin{itemize}
\tightlist
\item
  Needs way more color, pictures, graphics. I think there is too much text for this to be an effective way of ``selling'' the notebooks. You have to read a lot of prefaces before you get to the practical implementations and concrete examples.
\item
  Tone is kinda dry
\item
  Not sure if it's the layout, but I think just by nature of it being a book, it doesn't feel like as a user that I can or should skip around to only the sections I need. It's hard to know which sections will be most useful to me when I ``arrive'' to this resource.
\end{itemize}

\hypertarget{general-questions}{%
\chapter{General Questions}\label{general-questions}}

\begin{itemize}
\tightlist
\item
  What is the main takeaway from the

  tags (slide 25) in bookdown?
\end{itemize}

\hypertarget{outline-of-deliverables}{%
\chapter{Outline of Deliverables}\label{outline-of-deliverables}}

\hypertarget{how-to-make-packages}{%
\section{How to make packages}\label{how-to-make-packages}}

\begin{itemize}
\tightlist
\item
  \href{https://github.com/rstudio4edu/testpackage}{Demo package}
\item
  Companion tutorial \href{https://rstudio4edu.github.io/firstclasspackage/}{site} and \href{https://github.com/rstudio4edu/firstclasspackage}{repo}
\end{itemize}

\hypertarget{how-to-make-an-r-markdown-site-distill}{%
\section{How to make an R Markdown site / Distill}\label{how-to-make-an-r-markdown-site-distill}}

\begin{itemize}
\tightlist
\item
  Bare bones demo site - workshop
\item
  Blinged out demo site - workshop
\item
  Companion tutorial site - workshop

  \begin{itemize}
  \tightlist
  \item
    Divided into two parts, for bare bones with add-on for ``blinged out'' version?
  \end{itemize}
\item
  Bare bones demo site - course
\item
  Blinged out demo site - course
\item
  Compantion tutorial site - course
\end{itemize}

\hypertarget{deliverable}{%
\section{Deliverable}\label{deliverable}}

\hypertarget{deliverable-1}{%
\section{Deliverable}\label{deliverable-1}}

\hypertarget{timeline}{%
\chapter{Timeline}\label{timeline}}

Here we can put weekly goals to make sure we are staying on target.

Complete by the end of:

Week 12 (Aug 23):

Week 11 (Aug 16):

Week 10 (Aug 9):

Week 9 (Aug 2):

Week 8 (July 26):

Week 7 (July 19):

Week 6 (July 12):

Week 5 (July 5):

Week 4 (June 28):

Week 3 (June 21):

\bibliography{book.bib,packages.bib}


\end{document}
